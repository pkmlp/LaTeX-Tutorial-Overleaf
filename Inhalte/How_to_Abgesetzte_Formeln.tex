In diesem Kapitel wird gezeigt, wie mit abgesetzten Formeln \index{Formeln!abgesetzt} gearbeitet wird. Grössere mathematische Formeln oder Gleichungen setzt man besser in eigene Zeilen. Wenn keine Gleichungsnummer erstellt werden soll, stellt man sie zwischen \code{\textbackslash{begin}\{displaymath\}} und \code{\textbackslash{end}\{displaymath\}} oder zwischen \code{\textbackslash{[}} und \code{\textbackslash{]}}. Soll eine Gleichungsnummer erstellt werden, stellt man sie zwischen \code{\textbackslash{begin}\{equation\}} und \code{\textbackslash{end}\{equation\}}. 

\par\bigskip 
\textbf{\large Nicht} mehr verwenden sollte man \code{\$\$} ... \code{\$\$} für die Einbettung von abgesetzten Formeln!

Folgende Beispiele soll dies veranschaulichen:

\par\bigskip 
\par\bigskip 
Aus ...

\begin{tcolorbox}[width=\textwidth,colback={light-gray},title={Latex-Text},colbacktitle=gray,coltitle=white]

\begin{verbatim}
Seien $a$ und $b$ die Katheten und $c$ die
Hypotenuse, dann gilt: \[ a^2+b^2=c^2 \] Somit
gilt für die Hypothenuse: \[ c=\sqrt{a^2+b^2} \]
Lehrsatz des Pythagoras\index{Pythagoras}. 
\end{verbatim}

\end{tcolorbox}

... wird:

\begin{tcolorbox}[width=\textwidth,colback={light-gray},title={Print-Text},colbacktitle=gray,coltitle=white]

Seien $a$ und $b$ die Katheten und $c$ die
Hypotenuse, dann gilt: \[ a^2+b^2=c^2 \] Somit
gilt für die Hypothenuse: \[ c=\sqrt{a^2+b^2} \]
Lehrsatz des Pythagoras\index{Pythagoras}. 

\end{tcolorbox}

\textbf{Hinweis:} Mit\code{\textbackslash{index}\{Pythagoras\}} wird der Eintrag im Stichwort- / Namensverzeichnis gesetzt.


\pagebreak 
Sollen die Formeln abgesetzt (in einer eigenen Zeile zentriert freigestellt) werden, so kann man sie auch zwischen \code{\textbackslash{begin}\{displaymath\}} und \code{\textbackslash{end}\{displaymath\}} stellen:

Aus ...

\begin{tcolorbox}[width=\textwidth,colback={light-gray},title={Latex-Text},colbacktitle=gray,coltitle=white]

\begin{verbatim}
Seien $a$ und $b$ die Katheten und $c$ die
Hypotenuse, dann gilt: 

\begin{displaymath}
  a^2+b^2=c^2 
\end{displaymath}
 
Somit gilt für die Hypothenuse: 

\begin{displaymath}
  c=\sqrt{a^2+b^2} 
\end{displaymath}

Lehrsatz des Pythagoras\index{Pythagoras}. 
\end{verbatim}

\end{tcolorbox}

... wird:

\begin{tcolorbox}[width=\textwidth,colback={light-gray},title={Print-Text},colbacktitle=gray,coltitle=white]

Seien $a$ und $b$ die Katheten und $c$ die
Hypotenuse, dann gilt: 

\begin{displaymath}
  a^2+b^2=c^2 
\end{displaymath}
 
Somit gilt für die Hypothenuse: 

\begin{displaymath}
  c=\sqrt{a^2+b^2} 
\end{displaymath}

Lehrsatz des Pythagoras\index{Pythagoras}.  

\end{tcolorbox}

\textbf{Hinweis:} Mit\code{\textbackslash{index}\{Pythagoras\}} wird der Eintrag im Stichwort- / Namensverzeichnis gesetzt.


\pagebreak 
Sollen die Formeln nicht nur abgesetzt (in einer eigenen Zeile zentriert freigestellt), sondern auch nummeriert werden, so kann man sie auch zwischen \code{\textbackslash{begin}\{equation\}} und \code{\textbackslash{end}\{equation\}} stellen:

Aus ...

\begin{tcolorbox}[width=\textwidth,colback={light-gray},title={Latex-Text},colbacktitle=gray,coltitle=white]

\begin{verbatim}
Seien $a$ und $b$ die Katheten und $c$ die
Hypotenuse, dann gilt: 

\begin{equation}
  a^2+b^2=c^2 
\end{equation}
 
Somit gilt für die Hypothenuse: 

\begin{equation}
  c=\sqrt{a^2+b^2} 
\end{equation}

Lehrsatz des Pythagoras \index{Pythagoras}. 
\end{verbatim}

\end{tcolorbox}

... wird:

\begin{tcolorbox}[width=\textwidth,colback={light-gray},title={Print-Text},colbacktitle=gray,coltitle=white]

Seien $a$ und $b$ die Katheten und $c$ die
Hypotenuse, dann gilt: 

\begin{equation}
  a^2+b^2=c^2 
\end{equation}
 
Somit gilt für die Hypothenuse: 

\begin{equation}
  c=\sqrt{a^2+b^2} 
\end{equation}

Lehrsatz des Pythagoras \index{Pythagoras}. 

\end{tcolorbox}


\textbf{Hinweis:} Mit\code{\textbackslash{index}\{Pythagoras\}} wird der Eintrag im Stichwort- / Namensverzeichnis gesetzt.



\pagebreak 
Dies ermöglicht die Formeln später im Text zu referenzieren: \index{Formeln!referenziert}

Aus ...

\begin{tcolorbox}[width=\textwidth,colback={light-gray},title={Latex-Text},colbacktitle=gray,coltitle=white]

\begin{verbatim}
Seien $a$ und $b$ die Katheten und $c$ die
Hypotenuse, dann gilt: 

\begin{equation}
  a^2+b^2=c^2     \label{pythagoras:eins}
\end{equation}
 
Somit gilt für die Hypothenuse: 

\begin{equation}
  c=\sqrt{a^2+b^2}     \label{pythagoras:zwei}
\end{equation}

Lehrsatz des Pythagoras \index{Pythagoras}. Wobei 
\eqref{pythagoras:zwei} lediglich eine Umformung von 
\eqref{pythagoras:eins} ist.
\end{verbatim}

\end{tcolorbox}

... wird:

\begin{tcolorbox}[width=\textwidth,colback={light-gray},title={Print-Text},colbacktitle=gray,coltitle=white]

Seien $a$ und $b$ die Katheten und $c$ die
Hypotenuse, dann gilt: 

\begin{equation}
  a^2+b^2=c^2     \label{pythagoras:eins}
\end{equation}
 
Somit gilt für die Hypothenuse: 

\begin{equation}
  c=\sqrt{a^2+b^2}     \label{pythagoras:zwei}
\end{equation}

Lehrsatz des Pythagoras \index{Pythagoras}. Wobei 
\eqref{pythagoras:zwei} lediglich eine Umformung von 
\eqref{pythagoras:eins} ist.

\end{tcolorbox}


\textbf{Hinweis:} Mit\code{\textbackslash{index}\{Pythagoras\}} wird der Eintrag im Stichwort- / Namensverzeichnis gesetzt.
