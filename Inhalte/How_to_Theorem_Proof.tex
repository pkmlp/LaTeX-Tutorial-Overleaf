In diesem Kapitel wird gezeigt, wie mit Theorem und Proof gearbeitet wird. Dazu müssen in der Präambel des Dokumentes folgende Packages eingebunden werden:

\begin{tcolorbox}[width=\textwidth,colback={light-gray},title={Latex-Text},colbacktitle=gray,coltitle=white]

\begin{verbatim}
\usepackage{amsthm}
\newtheorem{theorem}{Satz}
\end{verbatim}

\end{tcolorbox}

\par\bigskip 
\par\bigskip 
Aus ...

\begin{tcolorbox}[width=\textwidth,colback={light-gray},title={Latex-Text},colbacktitle=gray,coltitle=white]

\begin{verbatim}
\begin{theorem}
Bei einem rechtwinkligen Dreieck mit den Seiten $a$, $b$, 
and $c$, gilt $a^2+b^2=c^2$. 
\end{theorem} 
\end{verbatim}

\end{tcolorbox}

... wird:

\begin{tcolorbox}[width=\textwidth,colback={light-gray},title={Print-Text},colbacktitle=gray,coltitle=white]

\begin{theorem}
Bei einem rechtwinkligen Dreieck mit den Seiten $a$, $b$, 
and $c$, gilt $a^2+b^2=c^2$. 
\end{theorem} 

\end{tcolorbox}


\par\bigskip 
\par\bigskip 
\par\bigskip 
Aus ...

\begin{tcolorbox}[width=\textwidth,colback={light-gray},title={Latex-Text},colbacktitle=gray,coltitle=white]

\begin{verbatim}
\begin{theorem}[Der Satz des Pythagoras] 
Bei einem rechtwinkligen Dreieck mit den Seiten $a$, $b$, 
and $c$, gilt $a^2+b^2=c^2$.
\end{theorem}
\end{verbatim}

\end{tcolorbox}

... wird:

\begin{tcolorbox}[width=\textwidth,colback={light-gray},title={Print-Text},colbacktitle=gray,coltitle=white]

\begin{theorem}[Der Satz des Pythagoras] 
Bei einem rechtwinkligen Dreieck mit den Seiten $a$, $b$, 
and $c$, gilt $a^2+b^2=c^2$.
\end{theorem}

\end{tcolorbox}



\pagebreak 
Aus ...

\begin{tcolorbox}[width=\textwidth,colback={light-gray},title={Latex-Text},colbacktitle=gray,coltitle=white]

\begin{verbatim}
\begin{theorem}[Der Satz des Pythagoras] 
Der Satz des Pythagoras ist einer der fundamentalen 
Sätze ... (aus Platzgründen gekürzt) ... 
\[
      a^{2}+b^{2}=c^{2}
\]
Der Satz ist nach Pythagoras von Samos benannt, der 
als Erster  ... (aus Platzgründen gekürzt) ...
\end{theorem}

\begin{proof}[\textbf{Beweis des Satzes des Pythagoras}]
Für den Satz sind mehrere hundert verschiedene 
Beweise bekannt. Der Satz des Pythagoras ist 
damit der meistbewiesene mathematische Satz.  
... (aus Platzgründen gekürzt) ... 
\end{proof}
\end{verbatim}

\end{tcolorbox}

... wird:

\begin{tcolorbox}[width=\textwidth,colback={light-gray},title={Print-Text},colbacktitle=gray,coltitle=white]

\begin{theorem}[Der Satz des Pythagoras] 
Der Satz des Pythagoras ist einer der fundamentalen 
Sätze ... (aus Platzgründen gekürzt) ... 
\[
      a^{2}+b^{2}=c^{2}
\]
Der Satz ist nach Pythagoras von Samos benannt, der 
als Erster  ... (aus Platzgründen gekürzt) ...
\end{theorem}

\begin{proof}[\textbf{Beweis des Satzes des Pythagoras}]
Für den Satz sind mehrere hundert verschiedene 
Beweise bekannt. Der Satz des Pythagoras ist 
damit der meistbewiesene mathematische Satz.  
... (aus Platzgründen gekürzt) ... 
\end{proof}

\end{tcolorbox}

