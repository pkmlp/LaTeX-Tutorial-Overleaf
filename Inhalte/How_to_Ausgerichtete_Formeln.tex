In diesem Kapitel wird gezeigt, wie Formeln ausgerichtet \index{Formeln!ausgerichtet} werden können. Sollen mehrere Formeln am Gleichheitszeichen ausgerichtet werden, muss das Zeichen \code{\&} am entsprechenden Ort gesetzt werden. 

Aus ...

\begin{tcolorbox}[width=\textwidth,colback={light-gray},title={Latex-Text},colbacktitle=gray,coltitle=white]

\begin{verbatim}
Da gibt es zwei Formeln, die jedes Kind kennt: 
\begin{align}
  a^2 + b^2 & = c^2     \label{pythagoras:drei}  \\
  e & = m c^2           \label{einstein:eins}  
\end{align}

Wobei \eqref{pythagoras:drei} Pythagoras\index{Pythagoras} 
und \eqref{einstein:eins} Albert Einstein\index{Einstein} 
zugeschrieben wird.
\end{verbatim}

\end{tcolorbox}

... wird:

\begin{tcolorbox}[width=\textwidth,colback={light-gray},title={Print-Text},colbacktitle=gray,coltitle=white]

Da gibt es zwei Formeln, die jedes Kind kennt: 
\begin{align}
  a^2 + b^2 & = c^2     \label{pythagoras:drei}  \\  
  e & = m c^2           \label{einstein:eins}  
\end{align}

Wobei \eqref{pythagoras:drei} Pythagoras\index{Pythagoras} 
und \eqref{einstein:eins} Albert Einstein\index{Einstein} 
zugeschrieben wird.

\end{tcolorbox}

\textbf{Hinweis:} Mit \code{\textbackslash{index}\{Pythagoras\}} und \code{\textbackslash{index}\{Einstein\}} werden die Einträge im Stichwort- / Namensverzeichnis gesetzt.
