Soll nun eine solche Abkürzung verwendet werden, geht das wie folgt.

\begin{tcolorbox}[width=\textwidth,colback={light-gray},title={Latex-Text},colbacktitle=gray,coltitle=white]

Ich bin angestellt an der \textcolor{red}{\textbackslash{ac\{eth\}}}, genauer an der \textcolor{red}{\textbackslash{ac\{ethz\}}}. Neben der \textcolor{red}{\textbackslash{ac\{ethz\}}} gibt es auch noch eine \textcolor{red}{\textbackslash{acs\{eth\}}} in der Westschweiz, nämlich die \textcolor{red}{\textbackslash{ac\{epfl\}}}. An der \textcolor{red}{\textbackslash{acs\{eth\}}} bin ich bei den \textcolor{red}{\textbackslash{ac\{id\}}} im Bereich \textcolor{red}{\textbackslash{ac\{pm\}}} tätig. Das \textcolor{red}{\textbackslash{ac\{pm\}}} ist eine Gruppe innerhalb der Sektion \textcolor{red}{\textbackslash{ac\{ppf\}}}.

\end{tcolorbox}


Die 'Norm' für die Verwendung von Abkürzungen sieht man mit der Verwendung der Abkürzung \textcolor{red}{\code{\textbackslash{ac\{ethz\}}}} an der zweiten und dritten Stelle im Text.


\begin{tcolorbox}[width=\textwidth,colback={light-gray},title={Print-Text},colbacktitle=gray,coltitle=white]

Ich bin angestellt an der \ac{eth}, genauer an der \ac{ethz}. Neben der \ac{ethz} gibt es auch noch eine \acs{eth} in der Westschweiz, nämlich die \ac{epfl}. An der \acs{eth} bin ich bei den \ac{id} im Bereich \ac{pm} tätig. Das \ac{pm} ist eine Gruppe innerhalb der Sektion \ac{ppf}.

\end{tcolorbox}

Am besten sieht man das, wenn im Latex Editor Texmaker die Tex-Datei geöffnet hat und man sich mit \keys{F1} die pdf Vorschau des Dokumentes anzeigen lässt.
