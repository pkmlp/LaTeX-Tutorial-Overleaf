Für die Verwendung von Fussnoten gelten folgende Regeln:


\begin{itemize} 
\item Fussnoten werden durch hochgestellte Ziffern nach dem Satzzeichen eines Absatzes bzw. Satzes oder nach einem zu zitierenden Wort angegeben. 

\item Mehrere Fussnoten an derselben Stelle sind nicht sinnvoll. 

\item Fussnoten gelten als ganze Sätze und müssen daher mit Grossbuchstaben begonnen und mit einem Punkt beendet werden. 

\item Gliederungsüberschriften dürfen nicht mit einer Fussnote versehen werden. 

\item Fussnoten gehören immer auf dieselbe Seite wie die Fussnotenreferenz im Text\footnote{Keine Regel ohne Ausnahme! In diesem Dokument werden alle Fussnoten am Ende des Dokumentes aufgeführt, um den Textfluss und das Seitenlayout nicht zu sehr zu stören.}. 

\item Ein Zitat aus einer anderen als der Originalquelle zu übernehmen (rezitieren) ist zu vermeiden.

\end{itemize}

Weiter sollte auch beachtet werden, dass die Arbeit auch ohne das Lesen der entsprechenden Fussnote verständlich sein muss. Daher gehören z.B. für die Argumentation wichtige Thesen nicht in eine Fussnote. Der in Fussnoten stehende Text sollte darum kurz und präzise sein. Exkurse sind zu vermeiden.
