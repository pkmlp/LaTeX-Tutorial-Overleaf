LaTeX (sprich: Latech) ist ein mächtiges Werkzeug zum Setzen von Schriftstücken. Gerade dann, wenn die Texte etwas länger werden, spielt es seinen entscheidenden Vorteil aus: Anders als andere Textverarbeitungssysteme stürzt es nie ab, auch bei mehr als 1000-seitigen Büchern nicht. Einer der weiteren grossen Vorteile von LaTeX ist es, dass es sich nach der Logik des Dokumentes richtet. LaTeX sagt man nicht \dq dieser Text hier soll etwas grösser und fett sein\dq{} sondern \dq dies hier ist eine Überschrift\dq. Alles andere erledigt LaTeX von selbst. LaTeX verarbeitet den Text teilweise vollautomatisch und nach einigen wenigen Anfangskonfigurationen kann man sich voll und ganz auf den Inhalt konzentrieren, statt sich mit der Formatierungen herumzuschlagen. Hinzu kommt, dass in LaTeX ein sehr leistungsfähiger, einfach zu bedienender Formeleditor eingebaut ist, was LaTeX vor allen Dingen für wissenschaftliche Arbeiten interessant macht. LaTeX wurde ausserdem mit dem Ziel geschrieben, dass Schriftstücke auch in 100 Jahren noch gleich aussehen und nicht so wie bei Textverarbeitungssytemen mal die Zeile auf die nächste Seite rutscht, mal die Bilder verrutschen oder gar nicht mehr da sind. LaTeX-Dokumente zu \dq programmieren\dq{} ist vergleichbar mit Internetseiten mit der Seitenbeschreibungssprache HTML zu programmieren. Es ist ähnlich einfach wie HTML und auch ähnlich aufgebaut. Jedoch ist es nicht für Internetseiten programmiert, sondern für beliebige Papierdokumente. Last but not least ist ein ausschlaggebendes Argument für LaTeX, dass es völlig kostenlos ist. 

Leider bringt LaTeX auch einen grossen Nachteil mit sich, den ich hier nicht verschweigen will. LaTeX muss man sozusagen \dq programmieren\dq. 

\begin{figure}[h!]
\centering
  \includegraphics[width=0.5\textwidth]{./Bilder/LaTeX_Anwender.jpg}
  \caption{Arbeiten mit \LaTeX}
\end{figure}

Einige Befehle sind zu erlernen. Wenn man einen gewissen Befehlssatz auswendig kann, dann gewinnt dieses \dq Programmieren\dq{} jedoch einen riesigen Geschwindigkeitszuwachs, so dass \dq normale \dq{} Text\-ver\-ar\-beit\-ungs\-systeme nicht mehr mithalten können. LaTeX ist auch nicht für hochgradige Designer-Texte geeignet. Der Designer kümmert sich um das Aussehen der Seite. Gerade dies ist bei LaTeX dem LaTeX-Übersetzer überlassen. 

LaTeX-Dokumente werden im Allgemeinen mittels einer Entwicklungsumgebung erstellt. Zwar kann man LaTeX-Dokumente auch mit Hilfe eines einfachen Texteditors erfassen und auf der Kommandozeile das Dokument erstellen. Doch bieten die auf LaTeX angepassten Programme mehr Funktionen und Komfort. Viele LaTeX-Befehle, Sonderzeichen und Symbole sind über die grafische Benutzeroberfläche zugänglich, und teilweise lassen sich darüber auch einfache Tabellen erstellen. Für grosse Projekte bieten Entwicklungsumgebungen eine Verwaltung und Strukturdarstellung. Dokumentenvorlagen und PDF-Vorschau finden sich fast überall. Zur Verbesserung der Lesbarkeit des Codes gibt es eine Syntax-Hervorhebung und teilweise auch eine Autovervollständigung von Befehlen. Manche Programme bieten eine Rechtschreibprüfung. Umlaute werden von manchen Entwicklungsumgebungen automatisch in LaTeX-Befehle übersetzt. Der fortschrittlichere Ansatz zur Verwendung von Umlauten ist eine geeignete Zeichenkodierung und ein dazu passendes LaTeX-Paket. Manche Umgebungen unterstützen insbesondere Unicode. 

Da LaTeX ein freies Produkt ist, welches auch im Sourcecode verfügbar ist, gibt es viele unterschiedliche sogenannte Distributionen. Das System wurde ehemals für den Unixbereich entwickelt, wurde aber auf sehr viele Systeme übertragen.

\textbf{MikTex\index{MikTex}} ist eine sehr gute LaTeX-Distribution für Windows. Während der Installation wird nachgefragt, wie mit fehelnden Packages umgegangen wird. Es ist empfohlen, diese mit Nachfrage installieren zu lassen, da bei der erstmaligen Übersetzung dieses Dokumentes werden so einige Packages nach-installiert. Diese sind jedoch für ein korrektes Layout und für die korrekte Funktionsweise notwendig. Der Installer dieser Distribution ist erhältlich unter  \code{\url{https://miktex.org/}}

\textbf{Texmaker\index{Texmaker}} ist ein plattformübergreifender Unicode-Texteditor für die Erstellung von LaTeX-Dokumenten. Die Software wird unter der GNU General Public License ver\-öf\-fent\-licht. Texmaker ist sehr gut bedienbar und übersichtlich. Ähnlich zu den anderen bekannten LaTeX - Editoren bietet auch er die grundlegende Unterstützung beim Einfügen von LaTeX-Konstruktionselementen - jedoch noch einige weitere Besonderheiten. Der Editor richtet sich insbesondere an LaTeX-Anfänger, denen mit Hilfe von Assistenten die Erstellung von Dokumenten erleichtert werden soll. Details und weiteres findet man unter \code{\url{http://www.xm1math.net/texmaker/}}

Dieses Tutorial/Template ist mit \textbf{Overleaf} erstellt worden. 
