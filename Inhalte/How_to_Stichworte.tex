
Es ist im Rahmen einer wissenschaftlichen Arbeit i.d.R. nicht unbedingt erforderlich, ein Stichwortverzeichnis zu erstellen. Als erstes jedoch ein Beispiel mit etwas Text, der lediglich dazu dient einige Einträge im Stichwortverzeichnis zu erzeugen.

Auf dieser Seite sind die Begriffe System\index{System}, Systemzustand\index{System!Zustand}, Systemelement\index{System!Element} und Emergenz\index{Emergenz} von Interesse. Auf dieser Seite sei nochmals was gesagt zu System\index{System} und Systemelement\index{System!Element} sowie zu Transformationsprozess.\index{Transformationsprozess}

\textbf{Wichtige Hinweise:}  
\begin{itemize}[rightmargin=1.0cm]
\item{Das Stichwortverzeichnis muss separat erstellt werden. In Texmaker erfolgt dies mit der Funktionstaste \keys{F12}.} 
\item{Das Standard-Stichwortverzeichnis ist meines Erachtens nicht schön. Darum habe ich für dieses eine eigene Stil-Datei \code{index.ist} erstellt. Diese muss in der Konfiguration von Texmaker angegeben werden (siehe auch \cref{fig:Konfig}: \nameref{fig:Konfig} auf Seite \pageref{fig:Konfig}).}
\end{itemize}
