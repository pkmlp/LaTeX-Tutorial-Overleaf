Arbeiten heisst immer auch dokumentieren. Projekt-, Produkt-, System-, Benutzer-Do\-ku\-men\-ta\-ti\-on\-en oder spätestens beim Abschluss der Berufslehre die IPA - das Erstellen von Dokumentationen ist (Berufs-) Alltag. Deshalb erhältst Du im Lehrlabor verschiedentlich die Aufgabe, eine schriftliche (technische, wissenschaftliche) Dokumentation zu verfassen.

Bevor nun mit der eigentlichen (fachlichen) Dokumentation begonnen wird:

\begin{quote}
    \begin{tcolorbox} 
        Vergewissere Dich unbedingt vor dem Beginn Deiner Arbeit bei Deiner Betreuungsperson, 
        ob sich die äussere Form (Formatierung, Layout, Reihenfolge der Kapitel, etc.) dieses
        Dokumentes mit den Vorgaben deckt, bzw. hole unbedingt bei Deiner Betreuungsperson ab, 
        an welche Richtlinien / Vorgaben Du Dich für Deine Dokumentation halten musst. 
    \end{tcolorbox}
\end{quote}

Das Template ist so aufgebaut, dass Anpassungen vorgenommen werden können, ohne sehr tiefe \LaTeX -Kenntnisse zu benötigen. Die Verwendung dieses Templates sollte somit keine Kopfschmerzen verursachen.  
