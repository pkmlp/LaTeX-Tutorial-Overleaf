In diesem Kapitel werden einige Beispiele von Formeln gezeigt.

\par\bigskip 
\par\bigskip 
\par\bigskip 
Aus ...

\begin{tcolorbox}[width=\textwidth,colback={light-gray},title={Latex-Text},colbacktitle=gray,coltitle=white]

\begin{verbatim}
\begin{equation} 
\int_0^{\infty} e^{-\rho} \rho^{2l}\left[ L_{n+l}^{2l+1} 
\left(\rho \right) \right]^2 \rho^2 d\rho = \frac{2n 
\left[\left(n+l\right)! \right]^3}{(n-l-1)!} 
\end{equation} 
\end{verbatim}

\end{tcolorbox}

... wird: 

\begin{tcolorbox}[width=\textwidth,colback={light-gray},title={Print-Text},colbacktitle=gray,coltitle=white]
\begin{equation} 
\int_0^{\infty} e^{-\rho} \rho^{2l}\left[ L_{n+l}^{2l+1} 
\left(\rho \right) \right]^2 \rho^2 d\rho = \frac{2n 
\left[\left(n+l\right)! \right]^3}{(n-l-1)!} 
\end{equation} 

\end{tcolorbox}

\par\bigskip 
\par\bigskip 
\par\bigskip 
Aus ...

\begin{tcolorbox}[width=\textwidth,colback={light-gray},title={Latex-Text},colbacktitle=gray,coltitle=white]

\begin{verbatim}
\[ 
\int_0^{\infty} e^{-\rho} \rho^{2l}\left[ L_{n+l}^{2l+1} 
\left(\rho \right) \right]^2 \rho^2 d\rho = \frac{2n 
\left[\left(n+l\right)! \right]^3}{(n-l-1)!} 
\] 
\end{verbatim}

\end{tcolorbox}

... wird: 

\begin{tcolorbox}[width=\textwidth,colback={light-gray},title={Print-Text},colbacktitle=gray,coltitle=white]
\[ 
\int_0^{\infty} e^{-\rho} \rho^{2l}\left[ L_{n+l}^{2l+1} 
\left(\rho \right) \right]^2 \rho^2 d\rho = \frac{2n 
\left[\left(n+l\right)! \right]^3}{(n-l-1)!} 
\]

\end{tcolorbox}
