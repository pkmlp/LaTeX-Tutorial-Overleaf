Alle Abkürzungen sind in einer separaten Datei mit Namen 'Abkuerzungen.tex' definiert. Diese Datei befindet sich im Verzeichnis 'Verzeichnisse' des Dokumentenordners.  Die Definitionen der Abkürzungen müssen mit folgendem Muster erstellt werden:
\begin{tcolorbox}[width=\textwidth,colback={light-gray},title={Erfassen von Abkürzungen},colbacktitle=gray,coltitle=white]
\begin{tabbing}
\code{\textbackslash{acro\{eth\}{[ETH]}\{Eidgenössische Technische Hochschule\}}}
\\
\small\code{\textbackslash{acro\{ethz\}{[ETHZ]}\{Eidgenössische Technische Hochschule Zürich\}}}
\\
\small\code{\textbackslash{acro\{epfl\}{[EPFL]}\{École polytechnique fédérale de Lausanne\}}}
\\
\small\code{\textbackslash{acro\{id\}{[ID]}\{Informatikdienste\}}}
\\
\small\code{\textbackslash{acro\{ppf\}{[PPF]}\{Procurement \& Portfolio Management\}}}
\\
\small\code{\indent\textbackslash{acro\{pm\}{[PM]}\{Portfolio Management\}}}
\end{tabbing}
\end{tcolorbox}
