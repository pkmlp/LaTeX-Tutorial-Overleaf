Dieses Dokument (erstellt mit \LaTeX\ - siehe Kapitel:  \nameref{Installation} auf Seite \pageref{Installation}), soll die Erstellung von umfangreichen technischen oder wissenschaftlichen Dokumentationen erleichtern.

\begin{tcolorbox}
Are you ready to leave those \dq what you see is what you get\dq{} word processors behind and to enter the world of real, reliable, and high-quality typesetting\cite{Kottwitz2011}?
\end{tcolorbox}

Ob in Uni, Beruf oder Alltag: Muss eine umfangreiche technische oder wissenschaftlich Dokumentation erstellt werden, führt über kurz oder lang kein Weg an \LaTeX\ vorbei.\par  

\textbf{Wofür eignet sich \LaTeX?}
\begin{compactitem}
\item jede Art wissenschaftlicher Veröffentlichungen 
\item Bücher (Sachbücher, Romane, Lexika, ...) 
\item Lebensläufe, Serienbriefe, Vorträge und Poster 
\item  ... 
\end{compactitem}
\par 

\textbf{Wofür eignet sich \LaTeX\ nicht!}
\begin{compactitem} 
\item Zeitungssatz  
\item Desktop Publishing (Plakate, Flyer etc.)  
\item sehr kurze Texte  
\item alle Bereiche, in denen Seiten-Elemente völlig frei angeordnet werden sollen
\item  ... 
\end{compactitem}
\par 

\LaTeX\ verführt immer wieder viele Anwender – insbesondere Anfänger – dazu, irgendetwas noch schöner zu machen: 'fancy' Schriften zu verwenden, am Layout herumzubasteln, Bilder punktgenau auf einer Seite zu platzieren, etc. Tatsächlich sind die Möglichkeiten nahezu unbegrenzt. Aber: Layout und Schriften sollen nicht schön sein, sondern den Inhalt der Arbeit aus Sicht des Lesers optimal transportieren.\par

\LaTeX\ ist somit nicht geeignet für jemanden, der sagt: 'Ich weiss selber am besten, wie mein Dokument aussehen soll.'. Man kann zwar mit \LaTeX\  im Prinzip alles machen, aber man muss entweder Glück haben und jemand anders hat schon ein entsprechendes Paket geschrieben, das man verwenden kann, oder man muss sehr viel Wissen über \TeX\ haben.\par

\LaTeX\ ist dagegen sehr gut geeignet für alle die sagen 'Ich möchte mich nur um den Inhalt, aber nicht um die Formatierung kümmern. Es soll aber trotzdem gut aussehen.'.\par 
