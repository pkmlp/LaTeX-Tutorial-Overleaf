In diesem Kapitel wird gezeigt, wie mit inline Formeln \index{Formeln!inline} gearbeitet wird. Bei Inline Formeln sind die mathmatichen Formeln im Fliesstext eingebettet (siehe auch Beispiel auf der vorhergehenden Seite).

\par\bigskip 
\par\bigskip 
\par\bigskip 
Folgendes Latex-Coding (beachte Abstände und normaler Text in Formel) ...

\begin{tcolorbox}[width=\textwidth,colback={light-gray},title={Latex-Text},colbacktitle=gray,coltitle=white]

\begin{verbatim}
$x^{2} \geq 0 \quad \textnormal{für alle} 
\quad x \in \mathbb{R}$
\end{verbatim}

\end{tcolorbox}

... wird wie folgt dargestellt: 

\begin{tcolorbox}[width=\textwidth,colback={light-gray},title={Print-Text},colbacktitle=gray,coltitle=white]

$x^{2} \geq 0 \quad \textnormal{für alle} 
\quad x \in \mathbb{R}$

\end{tcolorbox}


\par\bigskip 
\par\bigskip 
\par\bigskip 
\par\bigskip 
\par\bigskip 
Folgendes Latex-Coding (beachte doppelte Abstände und normaler Text in Formel) ...

\begin{tcolorbox}[width=\textwidth,colback={light-gray},title={Latex-Text},colbacktitle=gray,coltitle=white]

\begin{verbatim}
$x^{2} \geq 0 \qquad \textnormal{für alle} 
\qquad x \in \mathbb{R}$
\end{verbatim}

\end{tcolorbox}

... wird wie folgt dargestellt: 

\begin{tcolorbox}[width=\textwidth,colback={light-gray},title={Print-Text},colbacktitle=gray,coltitle=white]

$x^{2} \geq 0 \qquad \textnormal{für alle} 
\qquad x \in \mathbb{R}$

\end{tcolorbox}


\pagebreak
Folgendes Latex-Coding (beachte Formelkennzeichner) ...

\begin{tcolorbox}[width=\textwidth,colback={light-gray},title={Latex-Text},colbacktitle=gray,coltitle=white]

\begin{verbatim}
\(
x^{2} \geq 0 \qquad \textnormal{für alle} 
\qquad x \in \mathbb{R}
\)
\end{verbatim}

\end{tcolorbox}

... wird wie folgt dargestellt: 

\begin{tcolorbox}[width=\textwidth,colback={light-gray},title={Print-Text},colbacktitle=gray,coltitle=white]

\(
x^{2} \geq 0 \qquad \textnormal{für alle} 
\qquad x \in \mathbb{R}
\)

\end{tcolorbox}


\par\bigskip 
\par\bigskip 
\par\bigskip 
Folgendes Latex-Coding (beachte Grössse der Klammern) ...

\begin{tcolorbox}[width=\textwidth,colback={light-gray},title={Latex-Text},colbacktitle=gray,coltitle=white]

\begin{verbatim}
$ ((x+1) (x-1))^{2} $
\par\bigskip %um die Formeln besser abzugrenzen
$ \bigl( (x+1) (x-1) \bigr) ^{2} $
\par\bigskip %um die Formeln besser abzugrenzen
$\Bigl( \bigl( ( \quad x \quad ) \bigr) \Bigr)$
\par\bigskip %um die Formeln besser abzugrenzen
$ \Biggl( \biggl( \Bigl( \bigl( ( \quad x \quad ) 
   \bigr) \Bigr) \biggr) \Biggr) $
\end{verbatim}

\end{tcolorbox}

... wird wie folgt dargestellt: 

\begin{tcolorbox}[width=\textwidth,colback={light-gray},title={Print-Text},colbacktitle=gray,coltitle=white]

$ ((x+1) (x-1))^{2} $
\par\bigskip %um die Formeln besser abzugrenzen
$ \bigl( (x+1) (x-1) \bigr) ^{2} $
\par\bigskip %um die Formeln besser abzugrenzen
$\Bigl( \bigl( ( \quad x \quad ) \bigr) \Bigr)$
\par\bigskip %um die Formeln besser abzugrenzen
$ \Biggl( \biggl( \Bigl( \bigl( ( \quad x \quad ) 
  \bigr) \Bigr) \biggr) \Biggr) $

\end{tcolorbox}
