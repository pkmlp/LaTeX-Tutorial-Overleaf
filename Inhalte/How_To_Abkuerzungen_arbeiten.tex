Abkürzungen sollten in wissenschaftlichen Arbeiten, egal ob in Semester-, Prüfungs-, Bachelor- oder Masterarbeiten, möglichst spärlich und nur dann verwendet werden, wenn sie Klarheit und Lesbarkeit nicht beeinträchtigen. Dein angestrebter Leserkreis sollte sofort verstehen, was gemeint ist, oder es in einem vorangestellten Abkürzungsverzeichnis erklärt bekommen. 

Abkürzungen in Fachpublikationen sind i.d.R weniger Problematisch, sofern die verwendeten Abkürzungen im entsprechenden Fachbereich üblich sind.

Führe keine Abkürzung ein, die du nicht mindestens drei- oder viermal verwendest. Verwende stattdessen einfach den vollen Begriff.



