Um einen Begriff (oder Namen) in das Stichwortverzeichnis aufzunehmen, wird der entsprechende Begriff (oder Namen)  wie folgt markiert:

\begin{tcolorbox}[width=\textwidth,colback={light-gray},title={Latex-Text},colbacktitle=gray,coltitle=white]

Die Begriffe System\textcolor{red}{\textbackslash{index\{System\}}}, Systemzustand\textcolor{red}{\textbackslash{index\{System!Zustand\}}} und Systemelement\textcolor{red}{\textbackslash{index\{System!Element\}}} sollen im Stichwortverzeichnis aufgenommen werden.

\end{tcolorbox}

Im gedruckten Text sieht das dann wie folgt aus:

\begin{tcolorbox}[width=\textwidth,colback={light-gray},title={Print-Text},colbacktitle=gray,coltitle=white]

Die Begriffe System\index{System}, Systemzustand\index{System!Zustand} und Systemelement\index{System!Element} sollen im Stichwortverzeichnis aufgenommen werden.

\end{tcolorbox}

Im gedruckten Text ist nicht zu erkennen, ob ein bestimmter Begriff (oder Name) im Stichwortverzeichnis aufgenommen wird.
