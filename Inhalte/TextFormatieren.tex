An dieser Stelle sollen exemplarische einige Beispiele gezeigt werden, wie in \LaTeX\ Texte \textbf{Fett}, \textcolor{red}{farbig}, \textit{Kursiv}, \underline{unterstrichen}, \uuline{doppelt unterstrichen}, \uwave{unterschlängelt}, \sout{horizontal durchgestrichen}, \xout{schräg durchgestrichen} gesetzt werden und wie Texte formatiert / struk\-tu\-riert werden können.

Ebenso soll gezeigt werden, wie obige Beispiel-Aufzählung 

\begin{itemize}[topsep=0em, partopsep=0em, parsep=0em, itemsep=0em]
	\item \textbf{Fett}, 
	\item \textcolor{red}{farbig}, 
	\item \textit{Kursiv}, 
	\item \underline{unterstrichen}, 
	\item \uuline{doppelt unterstrichen}, 
	\item \uwave{unterschlängelt}, 
	\item \sout{horizontal durchgestrichen}, 
	\item \xout{schräg durchgestrichen},
\end{itemize}

auch als Liste gesetzt werden kann und wie ein selber definiertes Aufzählungszeichen 

\begin{itemize}[leftmargin=9em, topsep=0em, partopsep=0em, parsep=0em, itemsep=0em]
	\item[--]                 \textbf{Fett}, 
	\item[-]                  \textcolor{red}{farbig}, 
	\item[*]                  \textit{Kursiv}, 
	\item[\textgreater]       \underline{unterstrichen}, 
	\item[+]                  \uuline{doppelt unterstrichen}, 
	\item[zum Ersten:]        \uwave{unterschlängelt}, 
	\item[zum Zweiten:]       \sout{horizontal durchgestrichen}, 
	\item[als letztes noch:]  \xout{schräg durchgestrichen},
\end{itemize}

gesetzt werden kann.
